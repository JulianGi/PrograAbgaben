
\documentclass[a4paper]{article}

\usepackage{mathtools}
\usepackage{amsmath} % Advanced math typesetting
\usepackage{amssymb}
\usepackage[utf8]{inputenc} % Unicode support
\usepackage{graphicx}

\title{ Programmierung WS 18 \\ Hausaufgaben - Blatt 3 }
\author{ Julian Giesen (MNR 388487) \\
Levin Gäher (MNR 395035) \\
Gruppe 12 }
\date{  }

\addtolength{\textheight}{118pt} \addtolength{\voffset}{-59pt}
\addtolength{\textwidth}{60pt} \addtolength{\hoffset}{-30pt}

\begin{document}

\maketitle

\section*{ HA 2 }


Siehe Anhang

\section*{ HA 4 }

\subsection*{a)}
$\langle$ $0 \leq a.length$ $\rangle$\\
$\langle$ $0 \leq a.length \land 1=1 \land true = true \land 0 = 0$ $\rangle$\\
$x = 1;$ \\
$\langle$ $0 \leq a.length \land x=1 \land true = true \land 0 = 0$ $\rangle$\\
$result = true;$ \\
$\langle$ $0 \leq a.length \land x=1 \land result = true \land 0 = 0$ $\rangle$\\
$i = 0;$ \\
$\langle$ $i \leq a.length \land x=1 \land result = true \land i = 0$ $\rangle$\\
$\langle$ $i \leq a.length \land x=2^i \land result = \forall 0 \leq j < i: a[j] = 2^j$ $\rangle$\\
$while (i \leq a.length) \{$ \\
$\langle$ $i < a.length \land i \leq a.length \land x=2^i \land result = \forall 0 \leq j < i: a[j] = 2^j$ $\rangle$\\
$\langle$ $i + 1 \leq a.length \land 2x=2^{i + 1} \land result = \forall 0 \leq j < i: a[j] = 2^j$ $\rangle$\\
$if (a[i] != x) \{$\\
$\langle$ $a[i] != x \land i + 1 \leq a.length \land 2x=2^{i + 1} \land result = \forall 0 \leq j < i: a[j] = 2^j$ $\rangle$\\
$\langle$ $i + 1 \leq a.length \land 2x=2^{i + 1} \land false = \forall 0 \leq j < i + 1: a[j] = 2^j$ $\rangle$\\
$result = false;$\\
$\langle$ $i + 1 \leq a.length \land 2x=2^{i + 1} \land result = \forall 0 \leq j < i + 1 : a[j] = 2^j$ $\rangle$\\
$\}$\\
$\langle$ $i + 1 \leq a.length \land 2x=2^{i + 1} \land result = \forall 0 \leq j < i + 1: a[j] = 2^j$ $\rangle$\\
$x = x * 2$\\
$\langle$ $i + 1 \leq a.length \land x=2^{i + 1} \land result = \forall 0 \leq j < i + 1: a[j] = 2^j$ $\rangle$\\
$i = i + 1;$ \\
$\langle$ $i \leq a.length \land x=2^{i} \land result = \forall 0 \leq j < i: a[j] = 2^j$ $\rangle$\\
$\}$\\
$\langle$ $i \leq a.length \land x=2^{i} \land result = \forall 0 \leq j < i: a[j] = 2^j \land \neg (i < a.length)$ $\rangle$\\
$\langle$ $result = \forall 0 \leq a.length : a[j] = 2^j $ $\rangle$\\

\subsection*{b)}

Zur Verwendung der Bedingungsregel 1, muss Bewiesen werden, dass aus $result = \forall 0\leq j < i:a[j]2^j\land i + 1 \leq a.length$ und $\neg(a[i]!=x)$ die Nachbedingung $result = \forall 0 \leq j < i+1:a[j]=2^j\land i + 1 \leq a.length$ folgt.Da x sich nicht ändert, wenn die Gleichheit für ein weiteres Element überprüft wird - es wird nicht mehr bis i sondern i+1 verglichen - impliziert $result = \forall 0\leq j < i:a[j]2^j\land i + 1 \leq a.length$ und $\neg(a[i]!=x)$ die Aussage $result = \forall 0 \leq j < i+1:a[j]=2^j\land i + 1 \leq a.length$. Somit darf die Bedingungsregel 1 angewendet werden. \\\\
$V=a.length-i$\\
$B=>V\geq0$\\
$B=i<a.length => a.length-i \geq 0$
$\langle$ $a.length-i=m \land i<a.length$ $\rangle$\\
$\langle$ $a.length-(i+1) < m$ $\rangle$\\
$if(a[i]!=x)\{$\\
$\langle$ $a.length-(i+1)<m\land a[i]!=x$ $\rangle$\\
$\langle$ $a.length-(i+1)<m$ $\rangle$\\
$result = false;$\\
$\langle$ $a.length-(i+1)<m$ $\rangle$\\
$\}$\\
$\langle$ $a.length-(i+1)<m$ $\rangle$\\
$x=x*2$\\
$\langle$ $a.length-(i+1)<m$ $\rangle$\\
$i=i+1$\\
$\langle$ $a.length-i<m$ $\rangle$\\




\section*{ HA 6 }

	\includegraphics*[width=200px]{ProgramData_1.png} 
	\includegraphics*[width=200px]{ProgramData_2.png} 
	\includegraphics*[width=200px]{ProgramData_3.png} 
	\includegraphics*[width=200px]{ProgramData_4.png} 

\section*{ HA 8 }

Siehe Anhang

\end{document}
