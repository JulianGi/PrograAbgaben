

\documentclass[a4paper]{article}

\usepackage{mathtools}
\usepackage{amsmath} % Advanced math typesetting
\usepackage{amssymb}
\usepackage[utf8]{inputenc} % Unicode support
\usepackage{graphicx}
\usepackage{listings}
\usepackage{color}


\definecolor{dkgreen}{rgb}{0,0.6,0}
\definecolor{gray}{rgb}{0.5,0.5,0.5}
\definecolor{mauve}{rgb}{0.58,0,0.82}

\lstset{frame=tb,
  language=Java,
  aboveskip=3mm,
  belowskip=3mm,
  showstringspaces=false,
  columns=flexible,
  basicstyle={\small\ttfamily},
  numbers=none,
  numberstyle=\tiny\color{gray},
  keywordstyle=\color{blue},
  commentstyle=\color{dkgreen},
  stringstyle=\color{mauve},
  breaklines=true,
  breakatwhitespace=true,
  tabsize=3
}

\title{ Programmierung WS 18 \\ Hausaufgaben - Blatt 8 }
\author{ Julian Giesen (MNR 388487) \\
Levin Gäher (MNR 395035) \\
Gruppe 12 }
\date{  }

\addtolength{\textheight}{118pt} \addtolength{\voffset}{-59pt}
\addtolength{\textwidth}{60pt} \addtolength{\hoffset}{-30pt}

\begin{document}

\maketitle
\section*{HA 2 }

\section*{HA 4}
\subsection*{a)}
Die Gleichung stimmt nicht. Der $:$ Operator fügt die Liste links des Operators in die Liste rechts des Operators ein. Der $++$ Operator hängt die Liste links des Operators an die Liste rechts des Operators an.\\
$((x:[]) : []) : [] = [[[Int]]]$ 1 Element vom Typ $[[Int]]$\\
$((x : []) : []) ++ [] = [[Int]]$ 1 Element vom Typ $[Int]$
\subsection*{b)}
Die Gleichung Stimmt. Beide Listen enthalten $m+2$ Elemente vom Typ $Int$. 
\subsection*{c)}
Der Ausdruck $(x : y)$ ist syntaktisch inkorrekt, da $y$ keine Liste ist. Der Ausdruck $$(x : (y : (z : [])))$$ produziert eine Liste, die 3 Elemente vom Typ $Int$ enthält.
\subsection*{d)}
Die Gleichung stimmt. Beide Listen enthalten 2 Elemente vom Typ $[Int]$.
\subsection*{e)}
Die Gleichung stimmt nicht. Der linke Ausdruck fügt durch den $:$ Operator die Liste $[x,y]$ vom Typ $[Int]$ in die Liste $[[z]]$ vom Typ $[[Int]]$ ein, jedoch ohne $x,y$ und $z$ in einer Liste zu vereinen. Die neue Liste enthält zwei Elemente vom Typen $[Int]$. Der rechte Ausdruck fügt mit dem Operator $:$ zunächst $y$ in die Liste $[z]$, dann $x$ in die Liste $[y,z]$ und dann die Liste $[x,y,z]$ in die leere Liste $[]$ ein. Die resultierende Liste enthält somit ein Element vom Typen $[Int]$.
\section*{HA 6}
Siehe Anhang


\end{document}

% CODE:
% \begin{lstlisting}
% \end{lstlisting}
% 
% IMAGE:
% \includegraphics*[width=400px]{Nr6_UML_Graph.PNG}
