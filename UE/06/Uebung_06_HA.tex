
\documentclass[a4paper]{article}

\usepackage{mathtools}
\usepackage{amsmath} % Advanced math typesetting
\usepackage{amssymb}
\usepackage[utf8]{inputenc} % Unicode support
\usepackage{graphicx}
\usepackage{listings}
\usepackage{color}

\definecolor{dkgreen}{rgb}{0,0.6,0}
\definecolor{gray}{rgb}{0.5,0.5,0.5}
\definecolor{mauve}{rgb}{0.58,0,0.82}

\lstset{frame=tb,
  language=Java,
  aboveskip=3mm,
  belowskip=3mm,
  showstringspaces=false,
  columns=flexible,
  basicstyle={\small\ttfamily},
  numbers=none,
  numberstyle=\tiny\color{gray},
  keywordstyle=\color{blue},
  commentstyle=\color{dkgreen},
  stringstyle=\color{mauve},
  breaklines=true,
  breakatwhitespace=true,
  tabsize=3
}

\title{ Programmierung WS 18 \\ Hausaufgaben - Blatt 6 }
\author{ Julian Giesen (MNR 388487) \\
Levin Gäher (MNR 395035) \\
Gruppe 12 }
\date{  }

\addtolength{\textheight}{118pt} \addtolength{\voffset}{-59pt}
\addtolength{\textwidth}{60pt} \addtolength{\hoffset}{-30pt}

\begin{document}

\maketitle

\section*{ HA 2 }
\subsection*{a)}
(1)
A(int); explizit $->$ A(String); explizit$->$ Object(); implizit\\
Ausgabe: v1.x: written in A(int)\\
Begründung: x wird ausgegeben. Da der Konstruktor mit einem int aufgerufen wird, wird x in A(String) auf "written in A(int) gesetzt.
\\
(2)\\
Reihenfolge: B(int); explizit $->$ B(String); explizit $->$ A(String) explizit $->$ Object(); implizit\\ 
\\
Aufruf:System.out.println("v2.x: "+v2.x);\\
Ausgabe: v2.x: written in B(String)\\
Begründung: Wegen super() wird die Oberklasse von B aufgerufen. Deswegen wird A(String) aufgerufen.\\
\\
Aufruf: System.out.println("((B) v2).x: " + ((B) v2).x);\\
Ausgabe: ((B)v2).x: written in B(int)\\
Begründung: Das casten von A auf B sorgt dafür, dass das x von B über x von A priorisiert wird.\\
\\
(3)\\
Reihenfolge: B(A); explizit $->$ B(String); explizit $->$ A(String) $->$ Object(); implizit\\ \\
Aufruf: System.out.println("((A)v3).x: "+((A)v3).x);\\
Ausgabe: ((A)v3).x: written in B(String)\\
Begründung: v3 wird zu A gecastet. Somit wird die variable x aus A ausgegeben.\\
\\
Aufruf: System.out.println("v3.x: "+v3.x);\\
Ausgabe: v3.x: written in B(A)
Begründung: v3 ist instanceof B. Somit wird die variable x die in B liegt ausgegeben. 
\\
(4)\\
Reihenfolge: B() explizit $->$ B(String) explizit $->$ A(String) explizit $->$ Object(); \\
\\
Aufruf:System.out.println("((A)v4).x: "+((A)v4).x);
Ausgabe: ((A)v4).x: written in B(String)
Begründung: v4 wird zu A gecastet. Somit wird die variable x aus A ausgegeben.\\
\\
Aufruf: System.out.println("v4.x: "+v4.x);\\
Ausgabe: v4.x: written in B()
Begründung: Da v4 instanceof B ist, wird das x von B ausgegeben.

\subsection*{b)}
(1)Signatur: A.f(A)\\
(2)Signatur: A.f(A)\\
(3)Signatur: A.f(A)\\
(4)Signatur: B.f(A)\\
(5)Signatur: B.f(A)\\
(6)Signatur: B.f(A)\\
(7)Signatur: B.f(A)\\
(8)Signatur: B.f(A)\\
(9)Signatur: B.f(B)\\




\section*{ HA 5 }

\subsection*{Entry}

\begin{lstlisting}
public class Entry {
	private final String name;
	private final Node node;

	public Entry(String Name, Node Node) {
		name = Name;
		node = Node;
	}

	public String getName() {
		return name;
	}

	public Node getNode() {
		return node;
	}

	public File getAsFile() {
		if (node instanceof File)
			return (File) node;
		return null;
	}

	public Directory getAsDirectory() {
		if (node instanceof Directory)
			return (Directory) node;
		return null;
	}

	public Entry createHardlink(String newName) {
		return new Entry(newName, node);
	}
}
\end{lstlisting}

\subsection*{Node}

\begin{lstlisting}
import java.lang.*;

public abstract class Node {

	private long lastModified;

	public Node() {
		touch();
	}

	public long getLastModified() {
		return lastModified;
	}

	public void touch() {
		lastModified = System.currentTimeMillis();
	}
}
\end{lstlisting}

\subsection*{Directory}

\begin{lstlisting}
public class Directory extends Node {

	private Entry[] children;

	// Constructor

	public static Directory createEmpty() {
		return new Directory();
	}

	public Directory() {
		super();
		children = new Entry[0];
	}

	// General

	public Entry[] getEntries() {
		//return new Entry[children.length];
		return children;
	}

	public boolean containsEntry(String name) {
		return getEntry(name) != null;
	}

	public Entry getEntry(String name) {
		for (int i = 0; i < children.length; i++) {
			if (children[i].getName() == name)
				return children[i];
		}
		return null;
	}

	// Utility

	public void accept(String name, Visitor visitor) {
		visitor.visitDirectory(name, this);
		for (int i = 0; i < children.length; i++) {
			if (children[i].getNode() instanceof File)
				visitor.visitFile(children[i].getName(), (File) children[i].getNode());
			else if (children[i].getNode() instanceof Directory)
				((Directory) children[i].getNode()).accept(children[i].getName(), visitor);
		}
		visitor.visitedDirectory();
	}

	// Creators

	public Entry createDirectory(String name) {
		if (containsEntry(name)) {
			System.out.println("Error: Es existiert bereits einen Eintrag mit dem Namen'" + name + "'!");
			return null;
		}
		Entry dir = new Entry(name, new Directory());
		addEntry(dir);
		return dir;
	}

	public Entry createFile(String name, String content) {
		if (containsEntry(name)) {
			System.out.println("Error: Es existiert bereits einen Eintrag mit dem Namen'" + name + "'!");
			return null;
		}
		File f = new File();
		f.writeContent(content);
		Entry file = new Entry(name, f);
		addEntry(file);
		return file;
	}

	public Entry createHardlink(String name, Entry entry) {
		if (containsEntry(name)) {
			System.out.println("Error: Es existiert bereits einen Eintrag mit dem Namen'" + name + "'!");
			return null;
		}
		Entry link = new Entry(name, entry.getNode());
		addEntry(link);
		return link;
	}

	// Helpers

	private void addEntry(Entry entry) {
		Entry[] newChildren = new Entry[children.length + 1];
		for (int i = 0; i < children.length; i++)
			newChildren[i] = children[i];
		newChildren[children.length] = entry;
		children = newChildren;
	}
}
\end{lstlisting}

\subsection*{File}

\begin{lstlisting}
public class File extends Node {

	private String content;

	public File() {
		super();
	}

	public String readContent() {
		return content;
	}

	public void writeContent(String Content) {
		content = Content;
		this.touch();
	}

}
\end{lstlisting}

\subsection*{Visitor}

\begin{lstlisting}
public interface Visitor 
{
	void visitFile(String name, File file);
	void visitDirectory (String name, Directory directory);
	void visitedDirectory();
}
\end{lstlisting}

\subsection*{Printer}

\begin{lstlisting}
import java.lang.*;

public class Printer implements Visitor {
	public String curPath = "";

	public void visitFile(String name, File file) {
		System.out.println(file.getLastModified() + " " + curPath + name);
		System.out.println("> " + file.readContent());
	}

	public void visitDirectory(String name, Directory directory) {
		curPath = curPath + name + "/";
		System.out.println(directory.getLastModified() + " " + curPath);
	}

	public void visitedDirectory() {
		curPath = curPath.substring(0, curPath.lastIndexOf("/")) + "/";
	}
}
\end{lstlisting}


\end{document}
