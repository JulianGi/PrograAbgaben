
\documentclass[a4paper]{article}

\usepackage{mathtools}
\usepackage{amsmath} % Advanced math typesetting
\usepackage{amssymb}
\usepackage[utf8]{inputenc} % Unicode support
\usepackage{graphicx}

\title{ Programmierung WS 18 \\ Hausaufgaben - Blatt 6 }
\author{ Julian Giesen (MNR 388487) \\
Levin Gäher (MNR 395035) \\
Gruppe 12 }
\date{  }

\addtolength{\textheight}{118pt} \addtolength{\voffset}{-59pt}
\addtolength{\textwidth}{60pt} \addtolength{\hoffset}{-30pt}

\begin{document}

\maketitle

\section*{ HA 2 }
\subsection*{a)}
(1)
A(int); explizit $->$ A(String); explizit$->$ Object(); implizit\\
Ausgabe: v1.x: written in A(int)\\
Begründung: x wird ausgegeben. Da der Konstruktor mit einem int aufgerufen wird, wird x in A(String) auf "written in A(int) gesetzt.
\\
(2)\\
Reihenfolge: B(int); explizit $->$ B(String); explizit $->$ A(String) explizit $->$ Object(); implizit\\ 
\\
Aufruf:System.out.println("v2.x: "+v2.x);\\
Ausgabe: v2.x: written in B(String)\\
Begründung: Wegen super() wird die Oberklasse von B aufgerufen. Deswegen wird A(String) aufgerufen.\\
\\
Aufruf: System.out.println("((B) v2).x: " + ((B) v2).x);\\
Ausgabe: ((B)v2).x: written in B(int)\\
Begründung: Das casten von A auf B sorgt dafür, dass das x von B über x von A priorisiert wird.\\
\\
(3)\\
Reihenfolge: B(A); explizit $->$ B(String); explizit $->$ A(String) $->$ Object(); implizit\\ \\
Aufruf: System.out.println("((A)v3).x: "+((A)v3).x);\\
Ausgabe: ((A)v3).x: written in B(String)\\
Begründung: v3 wird zu A gecastet. Somit wird die variable x aus A ausgegeben.\\
\\
Aufruf: System.out.println("v3.x: "+v3.x);\\
Ausgabe: v3.x: written in B(A)
Begründung: v3 ist instanceof B. Somit wird die variable x die in B liegt ausgegeben. 
\\
(4)\\
Reihenfolge: B() explizit $->$ B(String) explizit $->$ A(String) explizit $->$ Object(); \\
\\
Aufruf:System.out.println("((A)v4).x: "+((A)v4).x);
Ausgabe: ((A)v4).x: written in B(String)
Begründung: v4 wird zu A gecastet. Somit wird die variable x aus A ausgegeben.\\
\\
Aufruf: System.out.println("v4.x: "+v4.x);\\
Ausgabe: v4.x: written in B()
Begründung: Da v4 instanceof B ist, wird das x von B ausgegeben.

\subsection*{b)}
(1)Signatur: A.f(A)\\
(2)Signatur: A.f(A)\\
(3)Signatur: A.f(A)\\
(4)Signatur: B.f(A)\\
(5)Signatur: B.f(A)\\
(6)Signatur: B.f(A)\\
(7)Signatur: B.f(A)\\
(8)Signatur: B.f(A)\\
(9)Signatur: B.f(B)\\




\section*{ HA 5 }

Siehe Anhang


\end{document}
